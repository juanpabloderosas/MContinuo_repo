\documentclass[12pt,a4paper]{article}

\sloppy

\setlength{\textwidth}{16cm}
\setlength{\textheight}{24cm}
\addtolength{\hoffset}{-1cm}
\addtolength{\voffset}{-2cm}

\usepackage[utf8]{inputenc}
\usepackage[spanish, es-tabla]{babel}
\usepackage{ae}
\usepackage{graphicx}

\usepackage{amsmath,amssymb,amsthm}
\usepackage{multicol, array}
\decimalpoint


\usepackage{sectsty}
%\allsectionsfont{\mdseries \raggedright}
\sectionfont{\fontsize{14}{15}\selectfont}
%\chapterfont{\large\sc\centering}
%\chaptertitlefont{\centering}
\subsectionfont{\fontsize{12}{15}\selectfont}
\subsubsectionfont{\fontsize{12}{15}\selectfont}


\usepackage[T1]{fontenc}
\usepackage{textcomp}
\usepackage[slantedGreek]{mathpazo} %% ver psnfss2e
\usepackage{pifont}
\usepackage{newcent}
\usepackage[small, bf, margin=20pt, tableposition=top]{caption}

\usepackage{fancyhdr}


\setlength{\parskip}{2mm}



\theoremstyle{definition}
\newtheorem{theorem}{Ejercicio N$^o$}

\pagestyle{fancy}
\lhead{Mecánica del Continuo 2023 \\  Licenciatura en Física}
\rhead{\includegraphics[width=1cm]{/home/juan/Documentos/Docencia/unsl.jpg}}
\vspace*{0.25cm}


\begin{document}

\begin{center}
\textbf{Trabajo Práctico N$^o$ 5: \\ Sólido Lineal Elástico}
\end{center}

\medskip

\section{Para hacer entre \textit{todes} - Clase TP5}
\subsection*{Muestra gratis}
Sabiendo que el acero tiene un límite elástico $t_A = 600 MPa$, se quiere diseñar una barra ($\mu = 82.0 GPa$ y $\lambda = 117.8 GPa$) de sección circular y de $5\, m$ de longitud, que sea capaz de sostener una masa $m=10^4 kg$, que trabaje dentro del régimen de sólido lineal elástico. También son requisitos que la elongación  de la varilla, al colgarse el peso, no supere los 3 cm y que el esfuerzo máximo de corte en cualquier plano no supere $0.8$ veces el límite elástico antes mencionado.

\vspace{0.25cm}

\noindent a) Calcular el radio mínimo de la barra que cumpla con los tres requisitos.

\noindent b) Calcular el radio después de colgar la masa requerida.


\section{Ejercicios para usted}
%EJERCICIO 1
 \begin{theorem}
El siguiente tensor representa el estado de esfuezo de un punto en un sólido: 
 
 \begin{center}
$[T] \; = \;
\left(\begin{smallmatrix}
1 & 2 & 3\\
2 & 4 & 5\\
3 & 5 & 0
\end{smallmatrix}\right)
MPa
$
 \end{center}

En cada uno de los planos normales a los vectores unitarios $\mathbf{e_1, e_2, e_3}$, (a) ¿cuál es el esfuerzo normal? y (b) ¿cuál es el esfuerzo tangencial total?

\end{theorem}

\bigskip

%\noindent \textbf{Ejercicio 1:} \newline

\begin{theorem}

Mostrar que si para un s\'olido el\'astico lineal, se cumple la relación 
\[
T_{ij}=\dfrac{\partial U}{\partial E_{ij}}
\]
\noindent luego se cumple la relaci\'on de simetr\'ia  $C_{ijkl}=C_{klij}$ para el tensor elasticidad.

\end{theorem}


%\noindent \textbf{Ejercicio 2:} \newline
\begin{theorem}

Utilizando la bibliografía:

\begin{itemize}

\item[a)] Derivar las expresiones del m\'odulo de Young, la raz\'on de Poisson, el módulo de volumen y el módulo de corte como funci\'on de los coeficientes de Lame.


\end{itemize}


\end{theorem}

%\noindent \textbf{Ejercicio 3:} \newline
\begin{theorem}

Realizar los ejercicios de la bibliografía (seg\'un nomenclatura de Lai $4^{a}$ edición): 5.4, 5.7, 5.8, 5.9, 5.10, 5.11, 5.13, 5.14, 5.17.

\end{theorem}


\pagebreak

%\noindent \textbf{Ejercicio 5:} \newline


\begin{theorem}

Una barra cilíndrica circular de sección A es sometida a estiramiento por una fuerza P en cada extremo.

(a) Determinar el esfuerzo normal y tangencial sobre un plano cuyo vector normal forma un ángulo $\alpha$ con el eje de la barra.

(b) ¿Para qué valor de $\alpha$ la magnitudes de esfuerzo normal y tangencial son iguales?

(c) Si la capacidad de carga de la barra está determinada por la condición de que el esfuerzo tangencial sobre un plano definido para $\alpha=\alpha_0$ sea menor que $\tau_0$, ¿cuál es el máximo valor posible de la fuerza P?


\end{theorem}


%\noindent \textbf{Ejercicio 6:} \newline

\begin{theorem}

Un eje cilíndrico de acero es sometido a torsión mediante un par de cuplas de 2700 $Nm$. El esfuerzo máximo permitido de estiramiento es de $0,124$ GPa.

(a) Derive el estado de esfuerzo en la barra.

(b) Si el esfuerzo tangencial máximo permitido es 0,6 veces el esfuerzo de estiramiento máximo, ¿cuál es el diámetro mínimo necesario para que el eje no se fracture?


\end{theorem} 

%\noindent \textbf{Ejercicio 4:} \newline
\begin{theorem}

Realizar los ejercicios de la bibliografía*  (seg\'un nomenclatura de Lai $4^{a}$ edición):5.38, 5.39, 5.41, 5.49.
Estos ejercicios son problemas elastodinámicos (ondas). Sólo para examen final.

\end{theorem}

\end{document}

