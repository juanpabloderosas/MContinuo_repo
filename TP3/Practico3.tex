\documentclass[11pt,a4paper]{article}

\sloppy

\setlength{\textwidth}{16cm}
\setlength{\textheight}{24cm}
\addtolength{\hoffset}{-1cm}
\addtolength{\voffset}{-2cm}

\usepackage[utf8]{inputenc}
%\usepackage[spanish, es-tabla]{babel}
\usepackage{ae}
\usepackage{graphicx}

\usepackage{amsmath,amssymb,amsthm}
\usepackage{multicol, array}
%\decimalpoint


\usepackage{sectsty}
%\allsectionsfont{\mdseries \raggedright}
\sectionfont{\fontsize{14}{15}\selectfont}
%\chapterfont{\large\sc\centering}
%\chaptertitlefont{\centering}
\subsectionfont{\fontsize{12}{15}\selectfont}
\subsubsectionfont{\fontsize{12}{15}\selectfont}


\usepackage[T1]{fontenc}
\usepackage{textcomp}
\usepackage[slantedGreek]{mathpazo} %% ver psnfss2e
\usepackage{pifont}
\usepackage{newcent}
\usepackage[small, bf, margin=20pt, tableposition=top]{caption}

\usepackage{fancyhdr}


\setlength{\parskip}{2mm}



\theoremstyle{definition}
\newtheorem{theorem}{Ejercicio N$^o$}

\pagestyle{fancy}
\pagestyle{fancy}
\lhead{Mecánica del Continuo 2023 -  U.N.S.L. \\ Depto. de Física -  Licenciatura en Física}
\rhead{\includegraphics[width=1cm]{unsl.jpg} }
\vspace*{0.25cm}


\begin{document}



\begin{center}
{\bf \large Práctico N$^o 3:$ \\
Mecánica del Continuo:}
\end{center}

\noindent \textbf{Temas:} descripción de la deformación, deformaciones infinitesimales, deformaciones finitas, Teorema de Descomposición Polar, Tensor de Cauchy-Green, tensor razón de cambio temporal de la deformación y tensor espín, Ecuación de Continuidad, cambio de elementos de área y volumen.

\medskip


%EJERCICIO1
\begin{theorem}

Mostrar que en un movimiento de rotación de cuerpo rígido en torno a un eje, descripto por:

\[
x-b  \: = \: R(t)(X-b)
\]

\noindent donde $R(t)$ es un tensor de rotación genérico, la distancia entre partículas permanece fija. 
\end{theorem}

\medskip


%EJERCICIO2
\begin{theorem}

Un cuerpo está rotando con una velocidad angular $\omega = \omega_1 \, e_1 + \omega_2 \, e_2 + \omega_3 \,e_3$.

\begin{itemize}
\item[\textbf{a)}] Encuentre el campo de velocidades asociado en coordenadas cartesianas.
\item[\textbf{b)}] Suponga $\omega = |\omega| e_3$. Evalúe las funciones que encontró en el punto (a).
\item[\textbf{c)}] Encuentre el campo de aceleración en coordenadas cartesianas para \linebreak $\omega = |\omega| \, e_3$.
\item[\textbf{d)}] ¿la velocidad de cada partícula varía en el tiempo? ¿cómo relaciona la aceleración con la pregunta anterior? 
\end{itemize}
\end{theorem}

\medskip

%EJERCICIO3
\begin{theorem}
Considere el movimiento dado por:
\bigskip
\begin{equation} \label{def}
x = X + X_1 \, k \, e_1
\end{equation}

Notamos que el tiempo desapareció en la ecuación anterior. Podemos considerar que: a) el tiempo no existe. b) el continuo se deforma desde una configuración incial,  llegando a un estado de equilibrio descrito por la ec. (\ref{def})....\textit{elija su propia aventura}... 

Sean $dX^{(1)} = (dS_1 / \sqrt{2}) (e_1 + e_2)$ y $dX^{(2)} = (dS_2 / \sqrt{2}) (-e_1 + e_2)$ elementos materiales diferenciales en la configuración no deformada. 

\begin{itemize}
\item[a)] Encuentre los elementos deformados según la ec. (\ref{def}), $dx^{(1)}$ y $dx^{(2)}$.
\item[b)] Evalúe las razones entre las longitudes $(ds^{(1)}-dS^{(1)})/dS^{(1)}$ y \linebreak $(ds^{(2)}-dS^{(2)})/dS^{(2)}$ ¿cómo son las razones?
\item[c)] Evalué el ángulo $\theta$,  entre los elementos no deformados ($dX^{(1)}$ y $dX^{(2)}$) y deformados ($dx^{(1)}$ y $dx^{(2)}$) del punto (b). ¿Cómo es el ángulo? Haga un esquema mostrando lo que acontece en las dos situaciones.
\item[d)] Mediante gráficas, compare los resultados de (b) y (c) con los obtenidos mediante el \textit{tensor de deformación infinitesimal} $E$ para  $0 < k < 1$.
\end{itemize}


\end{theorem}

\pagebreak



%Ejercicio 4
\begin{theorem}
 Considere el campo de deformación dado por $x = X + AX$, siendo $A$ un tensor de componentes $A_{ii}$ independientes de $X_i$.
 
\begin{itemize}
\item[a)] Realice gráficas en la forma del punto $(2.d)$ para un elemento material de su elección y para el ángulo $\theta$ entre dos elementos materiales --también a su elección-- comparando con el tensor $E$. Utilice valores $0 < A_{ii} < 0.5$
\item[b)] Agregue a las gráficas anteriores un tensor $E^A = (A+A^T)/2$ y compare. ¿para qué rango de $A_{ii}$ la aproximación es válida?
\end{itemize}

\end{theorem}

\medskip

%EJERCICIO5
\begin{theorem}

Realice los ejercicios a) 3.21 y b) 3.22 de la bibliografía Lai 4$^{ta}$ ed., ubicados en las páginas 147-148.

\end{theorem}

\medskip

%EJERCICIO6

\begin{theorem}
Dado el tensor de deformación infinitesimal:

\[
[E] \: = \: \left( \begin{array}{ccc}

5 & 3 & 0 \\
3 & 4 & -1 \\
0 & -1 & -2 \\

\end{array} \right)  
\times 10^{-4}
\] 

\begin{itemize}
\item[a)] Encuentre las direcciones de mayor deformación de un elemento infinitesimal.
\item[b)] Pruebe que el tensor $E$ de arriba no es el mismo que:

\[
[E'] \: = \: \left( \begin{array}{ccc}

3 & 0 & 0 \\
0 & 6 & 0 \\
0 & 0 & 2 \\

\end{array} \right)  
\times 10^{-4}
\] 


\end{itemize}

\end{theorem}


\medskip

\begin{theorem}
Realice los ejercicios 3.34, 3.37 de la bibliografía Lai 4$^{ta}$ ed., ubicados en las página 149.
\end{theorem}

\medskip

\begin{theorem}
Realice los ejercicios 3.38, 3.42, 3.44, 3.47, 3.48, 3.51, 3.52, 3.53 de la bibliografía Lai 4$^{ta}$ ed., ubicados en las páginas 149-150.
\end{theorem}

\medskip

%EJERCICIO 7	
\begin{theorem}
Dada la matriz del tensor de deformación infinitesimal:

\[
[E] \: = \: \left( \begin{array}{ccc}

k_1 X_1 & 0 & 0 \\
0 & -k_2 X_2 & 0 \\
0 & 0 & -k_2 X_2 \\

\end{array} \right)  
\] 

\begin{itemize}
\item[a)] Encuentre la ubicación de la(s) partícula que no cambia(n) su volumen.
\item[b)] ¿Cuál debería ser la relación entre $k_1$ y $k_2$ para que ninguna partícula cambie su volumen?
\end{itemize}
\end{theorem}


\medskip


\begin{theorem}
Realice los ejercicios 3.65, 3.68, 3.70 de la bibliografía Lai 4$^{ta}$ ed., ubicados en las páginas 151-152.
\end{theorem}

\medskip


%EJERCICIO7

\begin{theorem}
Verifique que, en coordenadas cartesianas, la deformación dada por:

\begin{align} \nonumber
x_1 = X_1 + kX_2 \; ; \; x_2 = X_2 \; ; \; x_3 = X_3
\end{align}

tiene un \textit{tensor derecho de elongación} $\mathbf{U}$ y un tensor de rotación $R$, dados por:


\begin{equation} \nonumber
[U] = \begin{pmatrix}
 f       &       kf/2     &  0\\
  kf/2 & (1+k^2/2)f &  0\\
   0    &         0       & 1
\end{pmatrix}
[R] = \begin{pmatrix} f & kf/2 & 0\\- kf/2 & f &  0\\ 0  & 0  & 1  \end{pmatrix}
\end{equation}
\noindent donde $f = (1+k^2/4)^{1/2}$
\end{theorem}



\end{document}

