\documentclass[12pt,a4paper]{article}

\sloppy

\setlength{\textwidth}{16cm}
\setlength{\textheight}{24cm}
\addtolength{\hoffset}{-1cm}
\addtolength{\voffset}{-2cm}

\usepackage[utf8]{inputenc}
%\usepackage[spanish, es-tabla]{babel}
\usepackage{ae}
\usepackage{graphicx}

\usepackage{amsmath,amssymb,amsthm}
\usepackage{multicol, array}


\usepackage{sectsty}
%\allsectionsfont{\mdseries \raggedright}
\sectionfont{\fontsize{14}{15}\selectfont}
%\chapterfont{\large\sc\centering}
%\chaptertitlefont{\centering}
\subsectionfont{\fontsize{12}{15}\selectfont}
\subsubsectionfont{\fontsize{12}{15}\selectfont}


\usepackage[T1]{fontenc}
\usepackage{textcomp}
\usepackage[slantedGreek]{mathpazo} %% ver psnfss2e
\usepackage{pifont}
\usepackage{newcent}
\usepackage[small, bf, margin=20pt, tableposition=top]{caption}

\usepackage{fancyhdr}


\setlength{\parskip}{2mm}



\theoremstyle{definition}
\newtheorem{example}{Ejercicio N$^o$}

\pagestyle{fancy}
\lhead{Mecánica del Continuo 2024 -  U.N.S.L. \\ Depto. de Física -  Licenciatura en Física}
\rhead{\includegraphics[width=1cm]{/home/juan/Documentos/Docencia/unsl.jpg}}
\vspace*{0.25cm}

\begin{document}

\begin{center}
{\textbf{Mecánica del Continuo} \\ 
 \textbf{Práctico N$^o$ 1:} Notación de índices, tensores, cálculo tensorial
}


\end{center}
\noindent \textbf{Nota:} Este práctico hay que entregarlo por escrito el jueves 28 de Marzo.
\medskip



\begin{example}
Problemas 2.1 al 2.18 de la bibliografía Lai (4a. edición)
\end{example}
\begin{center}
----------------------------------------------------------------------
\end{center}


\begin{example}

Problemas 2.19, 2.22, 2.24, 2.25, 2.26, 2.27, 2.28, \textbf{2.29}, \textbf{2.30}, 2.31, 2.32, 2.33,2.36, 2.37-(a,b), 2.39, 2.41, 2.42, 2.43,2.46, 2.48, 2.49, 2.51, 2.55, 2.58, 2.59, 2.60
de la bibliografía Lai (4a. edición). 

\noindent \textbf{En negrita:} estos ejercicios van a ser evaluados detalladamente. Incluir gráficas (a mano) de lo que considere necesario aclarar.

\end{example}
\begin{center}
----------------------------------------------------------------------
\end{center}

\begin{example}
Problemas 2.62, 2.63, 2.64, 2.65, 2.66, 2.67, 2.69, 2.70-d, 2.77, 2.78 de la bibliografía Lai (4ta. edición) .
\end{example}
\begin{center}
----------------------------------------------------------------------
\end{center}

\noindent \textbf{PS:}  para resolver los ejercicios deberá utilizar \textit{Mathematica} donde sea posible.



\end{document} 