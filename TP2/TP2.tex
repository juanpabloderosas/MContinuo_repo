\documentclass[11pt,a4paper]{article}

\sloppy

\setlength{\textwidth}{16cm}
\setlength{\textheight}{24cm}
\addtolength{\hoffset}{-1cm}
\addtolength{\voffset}{-2cm}

\usepackage[utf8]{inputenc}
%\usepackage[spanish]{babel}
\usepackage{ae}
\usepackage{graphicx}

\usepackage{amsmath,amssymb,amsthm}
\usepackage{multicol, array}


\usepackage{sectsty}
%\allsectionsfont{\mdseries \raggedright}
\sectionfont{\fontsize{14}{15}\selectfont}
%\chapterfont{\large\sc\centering}
%\chaptertitlefont{\centering}
\subsectionfont{\fontsize{12}{15}\selectfont}
\subsubsectionfont{\fontsize{12}{15}\selectfont}


\usepackage[T1]{fontenc}
\usepackage{textcomp}
\usepackage[slantedGreek]{mathpazo} %% ver psnfss2e
\usepackage{pifont}
\usepackage{newcent}
\usepackage[small, bf, margin=20pt, tableposition=top]{caption}

\usepackage{fancyhdr}


\setlength{\parskip}{2mm}



\theoremstyle{definition}
\newtheorem{example}{Ejercicio N$^o$}

\pagestyle{fancy}

\lhead{Mecánica del Continuo 2023 -  Lic. en Física \\ Departamento de Física -  U.N.S.L.}
\rhead{\includegraphics[width=1cm]{/home/juan/Documentos/Docencia/unsl.jpg}}
\vspace*{0.25cm}


\begin{document}



\begin{center}
{\bf \large Mecánica del continuo \\ Trabajo Práctico N$^o 2:$ }
\end{center}

\medskip

\noindent \textbf{Temas:} descripción del movimiento de un continuo - coordenadas materiales - descripción material y espacial del movimiento - velocidad y aceleración del continuo.



%EJERCICIO1
\begin{example}
Considerar el movimiento descripto por:
\center $x_1=kt+X_1, \quad x_2=X_2, \quad x_3=X_3$

\noindent donde la coordenada material especifica la posici\'on de la part\'icula a $t=0$.
\begin{itemize}

\item[a)] Determine la velocidad y aceleraci\'on de la part\'icula en la descripción material. Obtenga la trayectoria de las partículas cuyas coordenadas materiales están dadas por $X_1 = 0 m\, , X_2 \, , X_3 $

\item[b)] Determine los campos de velocidad y aceleración (descripción espacial). Grafique utilizando \textit{Mathematica} para $t = 0 \, s$ y para $t = 100 \, s$. 
 
\item[c)] Encuentre la razón temporal de cambio de la temperatura que experimenta una partícula, siendo que el campo de temperatura está especificado por $\Theta(x)=A x_1$. 

\end{itemize}

\end{example}

\center ----------------------------------------------------------------------


%EJERCICIO2
\begin{example}

Suponga que el movimiento de un continuo está descripto por $x_1=X_1, \quad x_2=X_2+kX_1^2t^2, \quad x_3=X_3$ y considere una zona, con forma de cuadrado unitario a $t=0$. 

\begin{itemize}

\item[a)] Determine la posición de los vértices a $t=1s$ y realice un esquema de al nueva forma de la figura.
\item[b)] Determine la velocidad y aceleración de una partícula del continuo.
\item[c)] Muestre que el campo de velocidad está dado por: 

\center $v_1=0, \quad v_3=0, \quad v_2=2 k x_1^2 t$.    
\end{itemize}
\end{example}

\center ----------------------------------------------------------------------


%EJERCICIO3
\begin{example}

La ecuación de movimiento que describe el movimiento de un continuo es $x_1=X_1-2 X_2^2 t^2, \quad x_2=X_2-X_3 t, \quad x_3=X_3$.

\begin{itemize}
\item[a)] Realice un esquema que describa el cambio en la forma de un segmento recto, cuyos extremos poseen coordenadas $(0,0,0)$ y $(0,1,0)$ en el instante de referencia.
\item[b)] Determine el valor de la velocidad a $t=2$ de la partícula cuyas coordenadas materiales son $(1,1,0)$ en el instante de referencia.
\item[c)] Encuentre la velocidad de la partícula que posee coordenadas $(1,1,0)$ cuando $t=2$.
\end{itemize}

\end{example}


\pagebreak

\center ----------------------------------------------------------------------
%EJERCICIO4
\begin{example}
Encuentre el campo de velocidad en la descripci\'on espacial del continuo cuya ecuaci\'on de movimiento es:

\begin{align*}
x_1=\dfrac{1+t}{1+t_0}X_1;  \quad x_2=X_2;  \quad x_3=X_3 
\end{align*}
\end{example}

\center ----------------------------------------------------------------------


%EJERCICIO5


\begin{example}
Considere el continuo cuyo campo de velocidad viene dado por:

\begin{equation} \nonumber
v_i=\dfrac{x_i}{1+t}
\end{equation}
\noindent Encuentre la ecuaci\'on de movimiento y el campo de aceleraci\'on en la descripci\'on material.
\end{example}

\center ----------------------------------------------------------------------

%EJERCICIO6
\begin{example}
Dado el campo de velocidad:

\center $v_1=-2x_2, \quad v_2=2x_1$

\begin{itemize}
\item[a)] Encuentre el campo de aceleraci\'on.
\item[b)] Obtenga la expresi\'on de la linea de camino (``pathline'') del continuo. 
\item[c)] Grafique el campo de aceleración en la descripción espacial.
\end{itemize}
\end{example}

\center ----------------------------------------------------------------------

%EJERCICIO7
\begin{example}
En la descripci\'on espacial, la ecuaci\'on para evaluar la aceleraci\'on es:
\begin{align*}
\dfrac{Dv}{Dt}=\left(\dfrac{\partial \mathbf{v}}{\partial t} \right)_x + (\nabla \mathbf{v}) \mathbf{v}
\end{align*}
\noindent Esta ecuación es no-lineal, es decir si consideramos dos campos de velocidad $v^A, \: v^B$, se cumple que:
\begin{align*}
a^A+a^B \neq a^{A+B} 
\end{align*}
\noindent donde $a^A$ y $a^B$ son los campos de aceleración de los campos de velocidad $v^A$ y $v^B$ (respectivamente) y $a^{A+B}$ es el campo de aceleración del campo de velocidad $v=v^A+v^B$.

Suponga un campo bidimensional dado por:

\begin{flalign*}
& v^A=-2x_2e_1+2x_1e_2 \\
& v^B=2x_2e_1-2x_1e_2 
\end{flalign*}

\begin{itemize}
\item[a)] Verifique esta desigualdad para los campos de velocidad anterior. Grafique $a^{A+B}$
\item[b)] Obtenga la suma $a^A + a^B$. Grafíquela.
\item[c)] ¿Qué condición debería cumplir $\nabla \mathbf{v}$ para que las gráficas anteriores coincidan?
\end{itemize}



\end{example}

%\center ----------------------------------------------------------------------

%EJERCICIO8
\begin{example}
Considerar el movimiento descripto por:
\begin{flalign*}
& x_1=X_1 \\
& x_2=X_2+ \text{sen}\left(\dfrac{\pi}{s} t\right) \text{sen}\left(\dfrac{\pi}{m} X_1\right) \\
& x_3=X_3 
\end{flalign*}
\noindent A $t=0$ un filamento material coincide con la linea recta que se extiende desde $(0,0,0)$ a $(1,0,0)$.
\begin{itemize}

\item[a)] Realice un esquema que muestre la forma de este filamento a $t=1/2s, \quad t=1s, \quad \text{y} \quad t=3/2s$.
\item[b)] Encuentre la velocidad y la aceleraci\'on en la descripci\'on espacial. 
\item[c)] Encuentre la velocidad y la aceleración en la descripción material.
\end{itemize}
\end{example}

----------------------------------------------------------------

\begin{example}
Considerar los campos de velocidad y  temperatura descriptos por:
\begin{flalign*}
& v=\dfrac{x_1e_1+x_2e_2}{x_1^2+x_2^2} \\
& \theta=k(x_1^2+x_2^2) 
\end{flalign*}

\begin{itemize}
\item[a)] Determine la velocidad en varias posiciones y describa la forma general de este campo de velocidad. ¿Qué forma poseen las isotermas? Grafique.
\item[b)] Determine la aceleraci\'on y la funci\'on que describe la raz\'on temporal de cambio de la temperatura de la part\'icula cuya coordenada material es $X=(1,1,0)m$. 
\end{itemize}
\end{example}





\end{document} 